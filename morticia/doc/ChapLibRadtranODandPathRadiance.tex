
\subsection{Transmittance and Path Radiance in libRadtran\label{sub:libRadtranRocks}}

REMs are required over the full range of SZA for all key conditions.
It is possible to compute the path transmittance between target and
sensor for the VZA equal to SZA (or the complement of SZA, since transmittance
is the same in either direction) by taking the ratio of the direct
solar irradiance at the lower height to the direct solar irradiance
at the upper height. Transmittance for paths at angles between the
SZA set will have to be interpolated. Horizontal paths are meaningless
in this context, since the pathlength is infinite. Curve-fitting to
the transmittance vector versus VZA should be considered as the relationship
will be non-linear. It is also very dependent on wavelength.

Once the transmittance has been computed for all sightlines in the
REM pair, the path radiance can be computed with reference to Figure
\ref{fig:Dual-Point-Radiance-Diagram}. Consider two points (generally
target and sensor) in the atmosphere at heights $h_{a}$ and $h_{b}$.
The two points lie on a linear slant path of a particular zenith angle.

\begin{figure}
\begin{centering}
$\,\,\,\,\xymatrix{ &  &  &  & \,\\
 & L_{ab}^{\uparrow},\tau_{ab}\ar@{-}[rr]\ar@{-}[dd] &  & b\ar[dl]_{L_{b}^{\downarrow}}\ar[ur]^{L_{b}^{\uparrow}} &  & h_{b}\\
 &  & \,\\
 & a\ar[dl]_{L_{a}^{\downarrow}}\ar[ur]^{L_{a}^{\uparrow}} &  & L_{ba}^{\downarrow}\ar@{-}[ll]\ar@{-}[uu] &  & h_{a}\ar[uu]\\
\, &  &  &  &  & \ar[u]
}
$
\par\end{centering}

\protect\caption{Dual-Point Radiance Diagram\label{fig:Dual-Point-Radiance-Diagram}}
\end{figure}


The radiance at point $a$ along the slant path in the downward direction
is $L_{a}^{\downarrow}$ and the upward radiance along the slant path
at point $a$ (towards point $b$) is denoted $L_{a}^{\uparrow}$
and likewise for point $b$. The path radiance in the upward direction
from $a$ to $b$ is $L_{ab}^{\uparrow}$ (the arrow is actually redundant
since the order of the subscripts can denote the direction) and the
downward path radiance is $L_{ba}^{\downarrow}$.The path transmittance
is denoted using $\tau$ with relevant subscripts. The path transmittance
is independent of the direction (but the path radiance does depend
on the direction). libRadtran can be used to compute the upward and
downward radiances along the slant path at each of the two altitudes.
The following equations are then easily solved for the path radiances
from the environment radiances and transmittances,

\begin{eqnarray}
L_{b}^{\uparrow} & = & \tau_{ab}L_{a}^{\uparrow}+L_{ab}^{\uparrow}\\
L_{a}^{\downarrow} & = & \tau_{ab}L_{b}^{\downarrow}+L_{ba}^{\downarrow}.
\end{eqnarray}


The path radiances are simply

\begin{align}
L_{ab}^{\uparrow} & =L_{b}^{\uparrow}-\tau_{ab}L_{a}^{\uparrow}\\
L_{ba}^{\downarrow} & =L_{a}^{\downarrow}-\tau_{ab}L_{b}^{\downarrow}.
\end{align}


Upward radiances are the radiance values in the lower hemispheres
of the REMs and downward radiances are radiance values in the upper
hemispheres of the REMs. The path radiances will generally comprise
a half hemisphere, above horizon and one side of the solar principal
plane for downwelling path radiances and sightlines looking above
horizon and another half hemisphere, below horizon and one side of
the solar principal plane for sightlines looking below the horizon.
These path radiances can be incorporated into a single dataset having
the same size as the REM. However, it is important to remember that
when looking upward, the target is at $h_{b}$ and is therefore rendered
using the REM at $h_{b}$ and when looking downward, the target is
at $h_{a}$ and must therefore be rendered using the REM at $h_{a}$. 

The at-target radiance is computed using a raytracing engine. The
camera image of the target can then be rendered using the BOSM and
inserted into the sensor view.

The way forward in reaching this implementation was first to establish
the runtime of a full radiant environment map with libRadtran using
the Kato correlated-$k$ parametrisation. The SZA resolution for a
specific atmosphere/surface combination is an important consideration
as path transmittance must be interpolated between computed SZA values.
On the Icebow compute cluster (24 processors), a full REM at 128 sightlines
per 180$^{\circ}$, with 8 Kato spectral bands, 2 heights and 12 SZAs
completes in under 1 minute. Compiling the results from the uvspec
output files takes a further 20 seconds. This means that the scheme
is certainly viable.

If, at this stage the raytracing engine cannot be used on the cluster
to generate correct azimuth/elevation views of the target, the above
approach can still be used to compute target illuminance, by integrating
the radiant environment map over the appropriate hemisphere in the
conventional way. This is an alternative to using SMARTS (and more
consistent, since the atmospheric models are now uniform across target
irradiance, path radiance and transmittance computations). 

If targets are completely lambertian and self-shadowing of the target
can be neglected, there are methods of reducing the REM to an Irradiance
Environment Map (IEM) and decomposing the IEM into a small number
of spherical harmonic coefficients. This allows very rapid rendering
of convex, lambertian targets. Computation of spherical harmonic coefficients
may have to be mediated through resampling to another directional
scheme, such as \href{http://healpix.jpl.nasa.gov/}{HEALPix}.

Thus, all of the radiative transfer could, in principle, be done using
libRadtran, which is a lot easier to set up than MODTRAN and also
faster, especially if the Kato correlated-$k$ approach is adopted.
The only limitation with Kato is that the spectral sensitivity of
the sensor can only be expressed as a linear combination of the bands
in Table \ref{tab:Kato-Correlated-k-Bands}. The libRadtran driver
utility (\textit{uvspec}) allows one to specify a sub-range of correlated-$k$
bands using the \textit{wavelength\_index} directive. Only 8 of the
32 Kato bands fall into the spectral region of interest in this problem.
These are bands 13 to 20, giving a spectral coverage of 605\,nm to
889\,nm.
